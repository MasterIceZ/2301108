\chapter{Convergence Test}

\section{Test for Divergence}

\begin{theorem}[Test for Divergence]
  If the series \(\sum_{n = 1}^{\infty} a_{n}\) is convergent, then \(\lim_{n \to \infty} a_{n} = 0\)
\end{theorem}

\begin{corollary}
  If \(\linf{n} a_{n}\) does not exist or \(\linf{n} a_{n} \neq 0\), then the series \(\sum_{n = 1}^{\infty} a_{n}\) is divergent.
\end{corollary}

\begin{lemma}
  If \(\sum a_{n}\) diverges and \(b_{n}\) converges, then \(\sum (a_{n} + b_{n})\) will be divergent.
\end{lemma}

\fbox{\begin{minipage}{\textwidth}
\begin{equation}
  \nonumber
  \begin{aligned}
    \sum_{n=1}^{\infty} \frac{2^{n} + 4^{n}}{e^n} &= \sum_{n = 1}^{\infty} \frac{2^{n}}{e^{n}} + \sinf{n=1} \frac{4^{n}}{e^{n}} \\
    &= \sum_{n=1}^{\infty} (\frac{2}{e})^{n} + (\frac{4}{e})^{n} \\
  \end{aligned}
\end{equation}
Since 
\begin{itemize}
  \item \(\frac{2}{e} < 1\) so \(\sum_{n=1}^{\infty} (\frac{2}{e})^{n}\) is convergent
  \item \(\frac{4}{e} > 1\) so \(\sum_{n=1}^{\infty} (\frac{4}{e})^{n}\) is divergent
\end{itemize}

Therefore, \(\sum_{n \to \infty} \frac{2^{n} + 4^{n}}{e^n}\) is divergent.
\end{minipage}}

\section{Integral Test}

\begin{theorem}[Integral Test]
  Let \(f(x)\) be a continuous, positive, and decreasing function on \([1, \infty)\). Then the series \(\sum_{n = 1}^{\infty} f(n)\) is convergent if and only if the improper integral \(\int_{1}^{\infty} f(x) \, dx\) is convergent.
  \begin{itemize}
    \item If \(int_{1}^{\infty} f(x) \, dx\) is convergent, then \(\sum_{n=1}^{\infty} a_{n}\) is convergent
    \item If \(int_{1}^{\infty} f(x) \, dx\) is divergent, then \(\sum_{n=1}^{\infty} a_{n}\) is divergent
  \end{itemize}
\end{theorem}

\begin{lemma}[P-series]
  The series \(\sum_{n=1}^{\infty} \frac{1}{n^p}\) is convergent if \(p > 1\) and divergent if \(p \leq 1\).
\end{lemma}

\fbox{\begin{minipage}{\textwidth}
\begin{equation}
  \nonumber
  \begin{aligned}
    \sum_{n=1}^{\infty} \frac{n}{n^4 + 1} \\
    f(x) &= \frac{x}{x^4 + 1} \\
    F(x) &= \int \frac{x}{x^4 + 1} \, dx \\
    &= \linf{t} \int_{1}^{t} \frac{x}{x^4 + 1} \, dx \\
  \end{aligned}
\end{equation}
Let $u = x^2$, then $du = 2x \, dx$
\begin{equation}
  \nonumber
  \begin{aligned}
    \linf{t} \int_{1}^{t} \frac{x}{x^4 + 1} \, dx &= \linf{t} \int_{x=1}^{x=t} \frac{x}{u^2+1} \cdot \frac{du}{2x} \\
    &= \linf{t} \frac{1}{2} \int_{1}^{\infty} \frac{1}{u^2 + 1} \, du \\
    &= \linf{t} \frac{1}{2} \arctan{u} \Big|_{1}^{t} \\
    &= \linf{t} \frac{1}{2} \arctan{t} - \frac{1}{2} \arctan{1} \\
    &= \frac{\pi}{4} - \frac{\pi}{8} \\
    &= \frac{\pi}{8}
  \end{aligned}
\end{equation}

\begin{equation}
  \nonumber
  \begin{aligned}
    \sum_{n=1}^{\infty} n(1 + n^2)^p \\
    f(x) &= x(1 + x^2)^p \\
    F(x) &= \int_{1}^{\infty} x(1 + x^2)^p \, dx \\
    &= \linf{t} \int_{1}^{t} x(1 + x^2)^p \, dx \\
  \end{aligned}
\end{equation}
Let $u = 1 + x^2$, then $du = 2x \, dx$
\begin{equation}
  \nonumber
  \begin{aligned}
    \linf{t} \int{1}^{t} x(1 + x^2)^p \, dx &= \linf{t} \int_{2}^{\infty} x \cdot u^p \, \frac{du}{2x} \\
    &= \linf{t} \frac{1}{2} \int_{2}^{\infty} u^p \, du \\
    &= \linf{t} \frac{1}{2} \frac{u^{p + 1}}{p + 1}\Big|_{2}^{\infty} \\
  \end{aligned}
\end{equation}

Since $p \neq -1$, then $p + 1 > 0$. Therefore, the integral is divergent.
\end{minipage}}

\section{Comparison Test}

\begin{theorem}[Comparison Test]
  Suppose that \(\sum a_{n}\) and \(\sum b_{n}\) are series with positive terms.
  \begin{itemize}
    \item If \(\sum b_{n}\) is convergent and \(a_{n} \leq b_{n}\) for all \(n\), then \(\sum a_{n}\) is convergent.
    \item If \(\sum b_{n}\) is divergent and \(a_{n} \geq b_{n}\) for all \(n\), then \(\sum a_{n}\) is divergent.
  \end{itemize}
\end{theorem}

In using the comparison test we must have some known series $\sum b_{n}$ for the purpose of comparison. Most of time we use one of these series:
\begin{itemize}
  \item P-series [\(\sum 1/n^{p}\) converges if $p > 1$ and diverges if $p \le 1$]
  \item Geometric series [\(\sum ar^{n-1}\) converges if $|r| < 1$ and diverges if $|r| \ge 1$]
\end{itemize}

\fbox{\begin{minipage}{\textwidth}
\begin{equation}
  \nonumber
  \begin{aligned}
    \sum_{n = 2}^{\infty} \frac{1}{\sqrt{n - 1}} \\
    \text{Let} \; a_{n} &= \frac{1}{\sqrt{n - 1}} \\
    \text{Let} \; b_{n} &= \frac{1}{\sqrt{n}} \\
    a_{n} &\geq b_{n} \; \text{for all} \; n \\
    \sum_{n = 2}^{\infty} \frac{1}{\sqrt{n}} &\leq \sum_{n = 2}^{\infty} \frac{1}{\sqrt{n - 1}} \\
    \sum_{n = 2}^{\infty} \frac{1}{\sqrt{n}} &\leq \sum_{n = 1}^{\infty} \frac{1}{\sqrt{n}} \\
  \end{aligned}
\end{equation}

Since \(\sum_{n = 1}^{\infty} \frac{1}{\sqrt{n}}\) is divergent, then \(\sum_{n = 2}^{\infty} \frac{1}{\sqrt{n - 1}}\) is divergent.
\end{minipage}}

\section{Limit Comparison Test}

\begin{theorem}[Limit Comparison Test]
  Suppose that \(\sum a_{n}\) and \(\sum b_{n}\) are series with positive terms. \\
  If
  \[
    \linf{n} \frac{a_{n}}{b_{n}} = L
  \]
  where \(L\) is a finite positive number and \(L > 0\), then either both series converge or both diverge.
\end{theorem}

\fbox{\begin{minipage}{\textwidth}
\[
  \sum_{n=1}^{\infty} \sin^{2}{(\frac{1}{n})}
\]
Choose $b_{n} = \frac{1}{n^{2}}$
\[
  \linf{n} \frac{sin^{2}{(\frac{1}{n})}}{\frac{1}{n^2}} = \linf{n} (\frac{\sin^{2}{\frac{1}{n}}}{\frac{1}{n}})^2
\]
Since $\linf{n} \frac{\sin{x}}{x} = 1$, then
\[
  \linf{n} (\frac{\sin^{2}{\frac{1}{n}}}{\frac{1}{n}})^2 = 1^2 = 1 > 0
\]

Since $\sum_{n = 1}{\infty} \frac{1}{n^2}$ converges. Then, by limit comparison test $\sum_{n=1}^{\infty} \sin^{2}{(\frac{1}{n})}$ converges.
\end{minipage}}

\fbox{\begin{minipage}{\textwidth}
Show that if $a_{n} > 0$ and $\linf{n} n \cdot a_{n} \neq 0$, then $\sum_{n=1}^{\infty} a_{n}$ is divergent.

\begin{proof}
  Since \(\linf{n} \frac{a_{n}}{\frac{1}{n}} = L\) where \(L > 0\) and \(\sum_{n=1}^{\infty} \frac{1}{n}\) diverges.
  By limit comparison test, \(\sum_{n=1}^{\infty} a_{n}\) is divergent.
\end{proof}
\end{minipage}}

\section{Alternating Series Test}

\begin{definition}[Alternating Series]
  An alternating series is a series of the form
  \[
    \sum_{n=1}^{\infty} (-1)^{n-1} b_{n} = b_{1} - b_{2} + b_{3} - b_{4} + \dots
  \]
  where \(b_{n} > 0\) for all \(n\).
\end{definition}

\begin{theorem}[Alternating Series Test]
  If the alternating series \(\sum_{n=1}^{\infty} (-1)^{n-1} b_{n}\) satisfies
  \begin{itemize}
    \item \(b_{n} \ge b_{n+1}\) for all \(n\)
    \item \(\linf{n} b_{n} = 0\)
  \end{itemize}
  then the series is convergent.
\end{theorem}

\fbox{\begin{minipage}{\textwidth}
\[
  \sinf{n=0} \frac{(-1)^{n + 1}}{\sqrt{n + 1}}
\]
Let $b_n = \Big|\frac{(-1)^{n + 1}}{\sqrt{n + 1}}\Big| =\frac{1}{\sqrt{n + 1}}$
1. Determine if the series is decreasing

From: 
\begin{equation}
  \nonumber
  \begin{aligned}
    \sqrt{n + 1} &\le \sqrt{(n + 1) + 1} \\ 
    \frac{1}{\sqrt{n + 1}} &\ge \frac{1}{\sqrt{n + 2}} \\
    b_{n} &\ge b_{n+1}
  \end{aligned}
\end{equation}

Since $b_{n} \ge b_{n + 1}$, the series is decreasing.

2. Determine if limit of $b_{n}$ is 0

\[
  \linf{n} \frac{1}{\sqrt{n + 1}} = 0
\]

Since the series is decreasing and the limit of $b_{n}$ is 0, the series is convergent.

Obervation:
\begin{equation}
  \nonumber
  \sinf{n=1} \frac{1}{\sqrt{n + 1}} = \sinf{n=1} \Big| \frac{(-1)^{n + 1}}{\sqrt{n + 1}} \Big|
\end{equation}
diverges.
\end{minipage}}