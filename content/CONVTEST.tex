\chapter{Convergence Test}

\section{Test for Divergence}

\begin{theorem}[Test for Divergence]
  If the series \(\sum_{n = 1}^{\infty} a_{n}\) is convergent, then \(\lim_{n \to \infty} a_{n} = 0\)
\end{theorem}

\begin{corollary}
  If \(\linf{n} a_{n}\) does not exist or \(\linf{n} a_{n} \neq 0\), then the series \(\sum_{n = 1}^{\infty} a_{n}\) is divergent.
\end{corollary}

\begin{lemma}
  If \(\sum a_{n}\) diverges and \(b_{n}\) converges, then \(\sum (a_{n} + b_{n})\) will be divergent.
\end{lemma}

\fbox{\begin{minipage}{\textwidth}
\begin{equation}
  \nonumber
  \begin{aligned}
    \sum_{n=1}^{\infty} \frac{2^{n} + 4^{n}}{e^n} &= \sum_{n = 1}^{\infty} \frac{2^{n}}{e^{n}} + \sinf{n=1} \frac{4^{n}}{e^{n}} \\
    &= \sum_{n=1}^{\infty} (\frac{2}{e})^{n} + (\frac{4}{e})^{n} \\
  \end{aligned}
\end{equation}
Since 
\begin{itemize}
  \item \(\frac{2}{e} < 1\) so \(\sum_{n=1}^{\infty} (\frac{2}{e})^{n}\) is convergent
  \item \(\frac{4}{e} > 1\) so \(\sum_{n=1}^{\infty} (\frac{4}{e})^{n}\) is divergent
\end{itemize}

Therefore, \(\sum_{n \to \infty} \frac{2^{n} + 4^{n}}{e^n}\) is divergent.
\end{minipage}}

\section{Integral Test}

\begin{theorem}[Integral Test]
  Let \(f(x)\) be a continuous, positive, and decreasing function on \([1, \infty)\). Then the series \(\sum_{n = 1}^{\infty} f(n)\) is convergent if and only if the improper integral \(\int_{1}^{\infty} f(x) \, dx\) is convergent.
  \begin{itemize}
    \item If \(int_{1}^{\infty} f(x) \, dx\) is convergent, then \(\sum_{n=1}^{\infty} a_{n}\) is convergent
    \item If \(int_{1}^{\infty} f(x) \, dx\) is divergent, then \(\sum_{n=1}^{\infty} a_{n}\) is divergent
  \end{itemize}
\end{theorem}

\begin{lemma}[P-series]
  The series \(\sum_{n=1}^{\infty} \frac{1}{n^p}\) is convergent if \(p > 1\) and divergent if \(p \leq 1\).
\end{lemma}

\fbox{\begin{minipage}{\textwidth}
\begin{equation}
  \nonumber
  \begin{aligned}
    \sum_{n=1}^{\infty} \frac{n}{n^4 + 1} \\
    f(x) &= \frac{x}{x^4 + 1} \\
    F(x) &= \int \frac{x}{x^4 + 1} \, dx \\
    &= \linf{t} \int_{1}^{t} \frac{x}{x^4 + 1} \, dx \\
  \end{aligned}
\end{equation}
Let $u = x^2$, then $du = 2x \, dx$
\begin{equation}
  \nonumber
  \begin{aligned}
    \linf{t} \int_{1}^{t} \frac{x}{x^4 + 1} \, dx &= \linf{t} \int_{x=1}^{x=t} \frac{x}{u^2+1} \cdot \frac{du}{2x} \\
    &= \linf{t} \frac{1}{2} \int_{1}^{\infty} \frac{1}{u^2 + 1} \, du \\
    &= \linf{t} \frac{1}{2} \arctan{u} \Big|_{1}^{t} \\
    &= \linf{t} \frac{1}{2} \arctan{t} - \frac{1}{2} \arctan{1} \\
    &= \frac{\pi}{4} - \frac{\pi}{8} \\
    &= \frac{\pi}{8}
  \end{aligned}
\end{equation}

\begin{equation}
  \nonumber
  \begin{aligned}
    \sum_{n=1}^{\infty} n(1 + n^2)^p \\
    f(x) &= x(1 + x^2)^p \\
    F(x) &= \int_{1}^{\infty} x(1 + x^2)^p \, dx \\
    &= \linf{t} \int_{1}^{t} x(1 + x^2)^p \, dx \\
  \end{aligned}
\end{equation}
Let $u = 1 + x^2$, then $du = 2x \, dx$
\begin{equation}
  \nonumber
  \begin{aligned}
    \linf{t} \int{1}^{t} x(1 + x^2)^p \, dx &= \linf{t} \int_{2}^{\infty} x \cdot u^p \, \frac{du}{2x} \\
    &= \linf{t} \frac{1}{2} \int_{2}^{\infty} u^p \, du \\
    &= \linf{t} \frac{1}{2} \frac{u^{p + 1}}{p + 1}\Big|_{2}^{\infty} \\
  \end{aligned}
\end{equation}

Since $p \neq -1$, then $p + 1 > 0$. Therefore, the integral is divergent.
\end{minipage}}

\section{Comparison Test}

\begin{theorem}[Comparison Test]
  Suppose that \(\sum a_{n}\) and \(\sum b_{n}\) are series with positive terms.
  \begin{itemize}
    \item If \(\sum b_{n}\) is convergent and \(a_{n} \leq b_{n}\) for all \(n\), then \(\sum a_{n}\) is convergent.
    \item If \(\sum b_{n}\) is divergent and \(a_{n} \geq b_{n}\) for all \(n\), then \(\sum a_{n}\) is divergent.
  \end{itemize}
\end{theorem}

In using the comparison test we must have some known series $\sum b_{n}$ for the purpose of comparison. Most of time we use one of these series:
\begin{itemize}
  \item P-series [\(\sum 1/n^{p}\) converges if $p > 1$ and diverges if $p \le 1$]
  \item Geometric series [\(\sum ar^{n-1}\) converges if $|r| < 1$ and diverges if $|r| \ge 1$]
\end{itemize}

\fbox{\begin{minipage}{\textwidth}
\begin{equation}
  \nonumber
  \begin{aligned}
    \sum_{n = 2}^{\infty} \frac{1}{\sqrt{n - 1}} \\
    \text{Let} \; a_{n} &= \frac{1}{\sqrt{n - 1}} \\
    \text{Let} \; b_{n} &= \frac{1}{\sqrt{n}} \\
    a_{n} &\geq b_{n} \; \text{for all} \; n \\
    \sum_{n = 2}^{\infty} \frac{1}{\sqrt{n}} &\leq \sum_{n = 2}^{\infty} \frac{1}{\sqrt{n - 1}} \\
    \sum_{n = 2}^{\infty} \frac{1}{\sqrt{n}} &\leq \sum_{n = 1}^{\infty} \frac{1}{\sqrt{n}} \\
  \end{aligned}
\end{equation}

Since \(\sum_{n = 1}^{\infty} \frac{1}{\sqrt{n}}\) is divergent, then \(\sum_{n = 2}^{\infty} \frac{1}{\sqrt{n - 1}}\) is divergent.
\end{minipage}}

\section{Limit Comparison Test}

\begin{theorem}[Limit Comparison Test]
  Suppose that \(\sum a_{n}\) and \(\sum b_{n}\) are series with positive terms. \\
  If
  \[
    \linf{n} \frac{a_{n}}{b_{n}} = L
  \]
  where \(L\) is a finite positive number and \(L > 0\), then either both series converge or both diverge.
\end{theorem}

\fbox{\begin{minipage}{\textwidth}
\[
  \sum_{n=1}^{\infty} \sin^{2}{(\frac{1}{n})}
\]
Choose $b_{n} = \frac{1}{n^{2}}$
\[
  \linf{n} \frac{sin^{2}{(\frac{1}{n})}}{\frac{1}{n^2}} = \linf{n} (\frac{\sin^{2}{\frac{1}{n}}}{\frac{1}{n}})^2
\]
Since $\linf{n} \frac{\sin{x}}{x} = 1$, then
\[
  \linf{n} (\frac{\sin^{2}{\frac{1}{n}}}{\frac{1}{n}})^2 = 1^2 = 1 > 0
\]

Since $\sum_{n = 1}{\infty} \frac{1}{n^2}$ converges. Then, by limit comparison test $\sum_{n=1}^{\infty} \sin^{2}{(\frac{1}{n})}$ converges.
\end{minipage}}

\fbox{\begin{minipage}{\textwidth}
Show that if $a_{n} > 0$ and $\linf{n} n \cdot a_{n} \neq 0$, then $\sum_{n=1}^{\infty} a_{n}$ is divergent.

\begin{proof}
  Since \(\linf{n} \frac{a_{n}}{\frac{1}{n}} = L\) where \(L > 0\) and \(\sum_{n=1}^{\infty} \frac{1}{n}\) diverges.
  By limit comparison test, \(\sum_{n=1}^{\infty} a_{n}\) is divergent.
\end{proof}
\end{minipage}}

\section{Alternating Series Test}

\begin{definition}[Alternating Series]
  An alternating series is a series of the form
  \[
    \sum_{n=1}^{\infty} (-1)^{n-1} b_{n} = b_{1} - b_{2} + b_{3} - b_{4} + \dots
  \]
  where \(b_{n} > 0\) for all \(n\).
\end{definition}

\begin{theorem}[Alternating Series Test]
  If the alternating series \(\sum_{n=1}^{\infty} (-1)^{n-1} b_{n}\) satisfies
  \begin{itemize}
    \item \(b_{n} \ge b_{n+1}\) for all \(n\)
    \item \(\linf{n} b_{n} = 0\)
  \end{itemize}
  then the series is convergent.
\end{theorem}

\fbox{\begin{minipage}{\textwidth}
\[
  \sinf{n=0} \frac{(-1)^{n + 1}}{\sqrt{n + 1}}
\]
Let $b_n = \Big|\frac{(-1)^{n + 1}}{\sqrt{n + 1}}\Big| =\frac{1}{\sqrt{n + 1}}$
1. Determine if the series is decreasing

From: 
\begin{equation}
  \nonumber
  \begin{aligned}
    \sqrt{n + 1} &\le \sqrt{(n + 1) + 1} \\ 
    \frac{1}{\sqrt{n + 1}} &\ge \frac{1}{\sqrt{n + 2}} \\
    b_{n} &\ge b_{n+1}
  \end{aligned}
\end{equation}

Since $b_{n} \ge b_{n + 1}$, the series is decreasing.

2. Determine if limit of $b_{n}$ is 0

\[
  \linf{n} \frac{1}{\sqrt{n + 1}} = 0
\]

Since the series is decreasing and the limit of $b_{n}$ is 0, the series is convergent.

Obervation:
\begin{equation}
  \nonumber
  \sinf{n=1} \frac{1}{\sqrt{n + 1}} = \sinf{n=1} \Big| \frac{(-1)^{n + 1}}{\sqrt{n + 1}} \Big|
\end{equation}
diverges.
\end{minipage}}

There are 3 ways to determine if an alternating series is convergent or divergent:
\begin{itemize}
  \item \(\sinf{n=1} (-1)^n \cdot b_n\) converges but \(\sinf{n=1} |(-1)^n\ cdot b_n|\) diverges
  \item \(\sinf{n=1} (-1)^n \cdot b_n\) converges and \(\sinf{n=1} |(-1)^n\ cdot b_n|\) converges (by comparison test)
  \item \(\sinf{n=1} (-1)^n \cdot b_n\) diverges and \(\sinf{n=1} |(-1)^n\ cdot b_n|\) diverges
\end{itemize}

\subsection{Estimating the Sum of an Alternating Series}

A partial sum $s_n$ of any convergent series can be used as an approximation to the total sum $s$. The error involved in using $s \approx s_n$ is the remainder $R_n = s - s_n$.

\begin{theorem}[Alternating Series Estimation Theorem]
  If \(s = \sinf{n=1} (-1)^{n-1} \cdot b_n\), where \(b_n > 0\), is the sum of an alternating series that satisfies
  \begin{itemize}
    \item \(b_{n} \ge b_{n+1}\) for all \(n\)
    \item \(\linf{n} b_{n} = 0\) (converges)
  \end{itemize}

  then \(|R_n| = |s - s_{n}| \le b_{n + 1}\)
\end{theorem}

\fbox{\begin{minipage}{\textwidth}
  Find the sum of the series \(\sinf{n = 0} \frac{(-1)^n}{n!}\) correct to three decimal places.

  Determine the series is convergent by the alternating series test.
  1. Determine if the series is decreasing
  \begin{equation}
    \nonumber
    \begin{aligned}
      b_{n + 1} &= \frac{1}{(n + 1)!} \\
      &= \frac{1}{n!(n + 1)} &&< b_{n} \\
      &&< \frac{1}{n!} 
    \end{aligned}
  \end{equation}
  2. Determine if limit of \(b_{n}\) is \(0\)
  \begin{equation}
    \nonumber
    \begin{aligned}
      \linf{n} \frac{1}{n!} &= 0
    \end{aligned}
  \end{equation}
  Such that the series is convergent.

  Use the alternating series estimation theorem to find the sum of the series correct to three decimal places.
  \begin{equation}
    \nonumber
    \begin{aligned}
      s &= \frac{1}{0!} - \frac{1}{1!} + \frac{1}{2!} - \frac{1}{3!} + \frac{1}{4!} - \frac{1}{5!} + \dots
      &= 1 - 1 + \frac{1}{2} - \frac{1}{6} + \frac{1}{24} - \frac{1}{120} + \frac{1}{720} - \frac{1}{5040} + \dots
    \end{aligned}
  \end{equation}
  Note: \(b_7 = \frac{1}{5040} \approx \frac{1}{5000} = 0.0002\)
  \[
    s_6 = 1 - 1 + \frac{1}{2} - \frac{1}{6} + \frac{1}{24} - \frac{1}{120} + \frac{1}{720} \approx 0.368056  
  \]

  By alternating series estimation theorem, \(|s - s_6| \le b_7 = 0.0002\)

  Therefore, the sum of the series is \(s \approx 0.368\)
\end{minipage}}

\begin{remark}
  For some \(\sinf{n=1} (-1)^n \cdot b_n\) converges but sometimes
  \begin{itemize}
    \item \(\sinf{n=1} |(-1)^n \cdot b_n|\) diverges (most of the time)
    \item \(\sinf{n=1} |(-1)^n \cdot b_n|\) converges
  \end{itemize}
  \(\sinf{n=1} (-1)^n \cdot b_n\) which is convergent relates with \(\sinf{n=1} |(-1)^n \cdot b_n|\) which is divergent.
\end{remark}

\begin{lemma}
  If we know that \(\pm \sinf{n=1} |(-1)^n \cdot b_n|\) converges, then \(\sinf{n=1} (-1)^n \cdot b_n\) also converges.
\end{lemma}

\begin{proof}
  \[
    - \sinf{n=1} |(-1)^n \cdot b_n| \le \sinf{n=1} (-1)^n \cdot b_n \le \sinf{n=1} |(-1)^n \cdot b_n|
  \]
  Since we know that \(\pm \sinf{n=1} |(-1)^n \cdot b_n|\) converges. Then by squeeze theorem, \(\sinf{n=1} (-1)^n \cdot b_n\) also converges.
\end{proof}

\section{Absolute Convergence and Conditional Convergence}

\begin{definition}[Absolute Convergence]
  A series \(\sinf{n=1} a_n\) is called absolutely convergent if the series of the absolute values \(\sinf{n=1} |a_n|\) is convergent.
\end{definition}

\begin{definition}[Conditional Convergence]
  A series \(\sinf{n=1} a_n\) is called conditionally convergent if the series is convergent but not absolutely convergent; that is, if \(\sinf{n=1} a_n\) is convergent but \(\sinf{n=1} |a_n|\) is divergent.
\end{definition}

\begin{theorem}
  If a series \(\sinf{n=1} a_n\) is absolutely convergent, then it is convergent.
\end{theorem}

\fbox{\begin{minipage}{\textwidth}
  \[
    \sinf{n=1} \sin{n}
  \]
  Consider:
  \[
    \sinf{n=1} |\sin{n}| = |\sin{1}| + |\sin{2}| + |\sin{3}| + \dots
  \]
  There are two cases
  \begin{itemize}
    \item If \(\sinf{n=1} |\sin{n}|\) converges, then \(\sinf{n=1} \sin{n}\) is convergent
    \item If \(\sinf{n=1} |\sin{n}|\) diverges, then we cannot conclude that \(\sinf{n=1} \sin{n}\) is divergent
  \end{itemize}
  In cases where \(\sinf{n=1} |a_n|\) diverges but \(\sinf{n=1} a_n\) converges, then \(\sinf{n=1} a_n\) is conditionally convergent.
\end{minipage}}

\fbox{\begin{minipage}{\textwidth}
  \[
    \sinf{n=1} \frac{\cos{n\pi}}{3n + 2}
  \]
  Since \(\cos{n\pi} = (-1)^n\), then
  \[
    \sinf{n=1} \frac{\cos{n\pi}}{3n + 2} = \sinf{n=1} \frac{(-1)^n}{3n + 2}
  \]
  Consider:
  \[
    \sinf{n=1} \Big| \frac{(-1)^n}{3n + 2} \Big| = \sinf{n=1} \frac{1}{3n + 2}
  \]
  \[
    \linf{n} \frac{\frac{1}{3n + 2}}{\frac{1}{n}} = \linf{n} \frac{n}{3n + 2} = \frac{1}{3} > 0
  \]
  and \(\sinf{n=1} \frac{1}{n}\) diverges from limit comparison test. Such that \(\sinf{n=1} \frac{1}{3n + 2}\) diverges. \\
  Consider: \(\sinf{n=1} \frac{(-1)^n}{3n + 2}\) which is an alternating series where \(b_n = \frac{1}{3n + 2}\)
  
  Let \(f(x) = \frac{1}{3x + 2}\)
  \[
    f'(x) = \frac{-3}{(3x + 2)^2} < 0
  \]
  So \(b_n\) is decreasing.

  \[
    \linf{n} \frac{1}{3n + 2} = 0
  \]
  Therefore, \(\sinf{n=1} \frac{(-1)^n}{3n + 2}\) is convergent.

  Since \(\sinf{n=1} \frac{(-1)^n}{3n + 2}\) is convergent and \(\sinf{n=1} \frac{1}{3n + 2}\) diverges, then \(\sinf{n=1} \frac{(-1)^n}{3n + 2}\) is conditionally convergent.
\end{minipage}}

\section{Ratio Test}

\begin{remark}
  This method might be good for series with factorials.
\end{remark}

This test is very useful in determining whether a given series is absolutely convergent.

\begin{theorem}[Ratio Test]
  There are three possible outcomes for the series \(\sinf{n=1} a_n\):
  \begin{itemize}
    \item If \(\linf{n} \Big| \frac{a_{n+1}}{a_n} \Big| = L < 1\), then the series \(\sinf{n=1} a_n\) is absolutely convergent.
    \item If \(\linf{n} \Big| \frac{a_{n+1}}{a_n} \Big| = L > 1\) or \(\linf{n} \Big| \frac{a_{n+1}}{a_n} \Big| = \infty\), then the series \(\sinf{n=1} a_n\) is divergent.
    \item If \(\linf{n} \Big| \frac{a_{n+1}}{a_n} \Big| = 1\), then the test is inconclusive.
  \end{itemize}
\end{theorem}

\fbox{\begin{minipage}{\textwidth}
  \[
    \sinf{n=1} \frac{n!}{100!}
  \]
  Consider:
  \begin{equation}
    \nonumber
    \begin{aligned}
      \linf{n} \Big| \frac{\frac{(n + 1)!}{100^{n + 1}}}{\frac{n!}{100^n}} &= \linf{n} \Big| \frac{(n + 1)! \cdot 100^n}{100^{n + 1} \cdot n!} \Big| \\
      &= \linf{n} \frac{n + 1}{100}
    \end{aligned}
  \end{equation}
  Since \(\linf{n} \frac{n + 1}{100} = \infty\), then the series \(\sinf{n=1} \frac{n!}{100!}\) is divergent.
\end{minipage}}

\begin{remark}
  Above example shows that factorial is bigger than exponential.
\end{remark}

\section{Root Test}

\begin{theorem}[Root Test]
  There are three possible outcomes for the series \(\sinf{n=1} a_n\):
  \begin{itemize}
    \item If \(\linf{n} \sqrt[n]{|a_n|} = L < 1\), then the series \(\sinf{n=1} a_n\) is absolutely convergent.
    \item If \(\linf{n} \sqrt[n]{|a_n|} = L > 1\) or \(\linf{n} \sqrt[n]{|a_n|} = \infty\), then the series \(\sinf{n=1} a_n\) is divergent.
    \item If \(\linf{n} \sqrt[n]{|a_n|} = 1\), then the test is inconclusive.
  \end{itemize}
\end{theorem}

\fbox{\begin{minipage}{\textwidth}
\[
  \sinf{n=1} \Big( \frac{1-n}{2+3n} \Big)^{n}
\]
Consider:
\begin{equation}
  \nonumber
  \begin{aligned}
    \linf{n} \sqrt[n]{\Big| \Big( \frac{1-n}{2+3n} \Big)^{n} \Big|} &= \linf{n} \Big| \frac{1-n}{2+3n} \Big| \\
    &= \linf{n} \frac{n - 1}{2 + 3n}
    &= \frac{1}{3} &&< 1
  \end{aligned}
\end{equation}
By root test: \(\sinf{n=1} \Big( \frac{1-n}{2+3n} \Big)^{n}\) is absolutely convergent.
\end{minipage}}

\fbox{\begin{minipage}{\textwidth}
  \[
    \sinf{n=1} \Big(\frac{-2n}{n + 1} \Big)^{5n}
  \]
  Consider:
  \begin{equation}
    \nonumber
    \begin{aligned}
      \linf{n} \sqrt[n]{\Big| \Big(\frac{-2n}{n + 1} \Big)^{5n} \Big|} &= \linf{n} \Big( \frac{2n}{n + 1} \Big)^{5} \\
      &= 2^{5} &&> 1
    \end{aligned}
  \end{equation}
  By root test: \(\sinf{n=1} \Big(\frac{-2n}{n + 1} \Big)^{5n}\) is divergent.
\end{minipage}}

\section{Strategy for Testing Series}

\begin{enumerate}
  \item \textbf{Test for divergence} \(\linf{n} a_{n} \neq 0\) \(\to\) divergent.
  \item \textbf{P-series} \(\frac{1}{n^p}\) converges if and only if \(p > 1\).
  \item \textbf{Geometric series} 
    \begin{itemize}
      \item find \(r\)
      \item converges if \(|r| < 1\) else diverges
    \end{itemize}
  \item \textbf{Comparison tests}
    \begin{itemize}
      \item \(a_n \le b_n\) and \(b_n\) converges such that \(a_n\) converges \(\equiv\) \(a_n \le b_n\) and \(a_n\) diverges such that \(b_n\) diverges
      \item Limit comparison \\
      If
      \[
        \linf{n} \frac{a_n}{b_n} = C > 0
      \]
      then
      \[
        \sinf{n=1} a_n \, \mathrm{converges} \Leftrightarrow \sinf{n=1} b_n \, \mathrm{converges} \equiv \sinf{n=1} a_n \, \mathrm{diverges} \Leftrightarrow \sinf{n=1} b_n \, \mathrm{diverges}
      \]
    \end{itemize}
  \item \textbf{Alternating series test}
    \begin{itemize}
      \item Find real \(b_n\) and \(\mathrm{sgn}{(a_n)}\)
      \item Determine if \(b_n\) is decreasing \(b_{n + 1} \le b_{n}\)
      \item Determine if \(\linf{n} b_n = 0\)
    \end{itemize}
    If satisfies all conditions, then the series is convergent.
  \item \textbf{Ratio test}
    \[
      \linf{n} \Big| \frac{a_{n+1}}{a_n} \Big| = L \begin{cases}
        < 1 & \text{absolutely convergent} \\
        > 1 & \text{divergent} \\
        = 1 & \text{inconclusive}
      \end{cases}
    \]
  \item \textbf{Root test}
    \[
      \linf{n} \sqrt[n]{|a_n|} = L \begin{cases}
        < 1 & \text{absolutely convergent} \\
        > 1 & \text{divergent} \\
        = 1 & \text{inconclusive}
      \end{cases}
    \]
  \item \textbf{Integral test} \\
  If \(f(x)\) is continuous, positive, and decreasing on \([1, \infty)\), then
  \[
    \int_{1}^{\infty} f(x) \, dx \, \mathrm{converges} \Leftrightarrow \sinf{n=1} f(n) \, \mathrm{converges}
  \]
\end{enumerate}