\chapter{Partial Derivatives}

\section{Functions of Several Variables}

\begin{definition}[Function of Two Variables]
  A function \(f\) of two variables is a rule that assigns to each ordered pair \((x, y)\) in a set \(D\) a unique real number denoted by \(f(x, y)\). \\
  The set \(D\) is the domain of \(f\) and the set of all possible values of \(f(x, y)\) is the range of \(f\).
\end{definition}

\begin{definition}[Graphs]
  If \(f\) is a function of two variables with domain \(D\), then the graph of \(f\) is the set of all points \((x, y, z)\) in space where \(z = f(x, y)\).
\end{definition}

\begin{definition}[Level Curves]
  The level curves of a function \(f\) of two variables are the curves with equations \(f(x, y) = k\) where \(k\) is a constant (in range of \(f\)).
\end{definition}

\section{Limits and Continuity}

\subsection{Limits of Functions of Two Variables}

Limits of functions of two variables are written as
\[
  \lim_{(x, y) \to (a, b)} f(x, y) = L
\]

\begin{definition}[Limit of a Function of Two Variables]
  Let \(f\) be a function of two variables defined on some open region that includes points arbitrarily close to \((a, b)\) except possibly at \((a, b)\) itself. Then
  \[
    \lim_{(x, y) \to (a, b)} f(x, y) = L
  \]
  if for every number \(\epsilon > 0\), there exists a number \(\delta > 0\) such that
  \[
    |f(x, y) - L| < \epsilon
  \]
  whenever \((x, y)\) is in the domain of \(f\) and satisfies
  \[
    0 < \sqrt{(x - a)^2 + (y - b)^2} < \delta
  \]
\end{definition}

\begin{theorem}[Limit Existance]
  If \(f(x, y) \to L_1\) as \((x, y) \to (a, b)\) along a path \(C_1\) and \(f(x, y) \to L_2\) as \((x, y) \to (a, b)\) along a path \(C_2\), then \(L_1 = L_2\).
\end{theorem}

\begin{corollary}[Limit not Exists]
  If \(f(x, y)\) approaches different values as \((x, y) \to (a, b)\) along different paths, then the limit does not exist.
\end{corollary}

\fbox{\begin{minipage}{\textwidth}
  Show that the limit does not exist.
  \[
    \lim_{(x, y) \to (0, 0)} \frac{2xy}{x^2 + 3y^2}
  \]
  Consider path \(C_1\) where \(y = 0\)
  \begin{equation}
    \nonumber
    \begin{aligned}
      \lim_{(x, y) to (0, 0)} \frac{2xy}{x^2 + 3y^2} &= \lim_{x \to 0} \frac{2x \cdot 0}{x^2 + 3 \cdot 0^2} \\
      &= \lim_{(x, y) \to (0, 0)} 0 \\
      &= 0
    \end{aligned}
  \end{equation}
  Consider path \(C_2\) where \(x = y\)
  \begin{equation}
    \nonumber
    \begin{aligned}
      \lim_{(x, y) to (0, 0)} \frac{2xy}{x^2 + 3y^2} &= \lim_{x \to 0} \frac{2x^2 \cdot 0}{4x^2} \\
      &= \lim_{(x, y) \to (0, 0)} \frac{2}{4} \\
      &= \frac{1}{2}
    \end{aligned}
  \end{equation}
  Since the limit approaches different values along different paths, the limit does not exist.
\end{minipage}}

\subsubsection{Properties of Limits}

\textbf{Sum Law} Limit of a sum is the sum of limits. \\
\textbf{Difference Law} Limit of a difference is the difference of limits. \\
\textbf{Constant Multiple Law} Limit of a constant multiple is the constant multiple of the limit. \\
\textbf{Product Law} Limit of a product is the product of limits. \\
\textbf{Quotient Law} Limit of a quotient is the quotient of limits. \\

\fbox{\begin{minipage}{\textwidth}
  Use the Squeeze Theorem to find the limit.
  \[
    \lim_{(x, y) \to (0, 0)} xy \sin{\frac{1}{x^2+y^2}}
  \]
  By the Squeeze Theorem, we have
  \begin{equation}
    \nonumber
    \begin{aligned}
      -1 &\le \sin{\frac{1}{x^2+y^2}} &&\le 1 \\
      -xy &\le xy \sin{\frac{1}{x^2+y^2}} &&\le xy
    \end{aligned}
  \end{equation}
  Since 
  \[
    \lim_{(x, y) \to (0, 0)} -xy = 0
  \]
  and 
  \[
    \lim_{(x, y) \to (0, 0)} xy = 0
  \]
  By the Squeeze Theorem, the limit is 0.
\end{minipage}}

\fbox{\begin{minipage}{\textwidth}
  Use the Squeeze Theorem to find the limit.
  \[
    \lim_{(x, y) \to (0, 0)} \frac{xy^4}{x^4 + y^4}
  \]
  By the Squeeze Theorem, we have
  \begin{equation}
    \nonumber
    \begin{aligned}
      y^4 &\le x^4 + y^4 \\
      \frac{y^4}{x^4 + y^4} &\le 1 \\
      -1 &\le \frac{y^4}{x^4 + y^4} \le 1 \\
      -x &\le \frac{xy^4}{x^4 + y^4} \le x
    \end{aligned}
  \end{equation}
  Since
  \[
    \lim_{(x, y) \to (0, 0)} -x = 0
  \]
  and
  \[
    \lim_{(x, y) \to (0, 0)} x = 0
  \]
  By the Squeeze Theorem, the limit is 0.
\end{minipage}}

\subsection{Continuity}

\begin{definition}[Continuity]
  A function \(f\) of two variables is continuous at a point \((a, b)\) if
  \[
    \lim_{(x, y) \to (a, b)} f(x, y) = f(a, b)
  \]
\end{definition}

\fbox{\begin{minipage}{\textwidth}
  Determine the set of points at which the function is continuous.
  \[
    f(x, y) = \begin{cases}
      \frac{x^2y^3}{2x^2+y^2} & \text{if } (x, y) \neq (0, 0) \\
      0 & \text{if } (x, y) = (0, 0)
    \end{cases}
  \]
  Consider
  \begin{equation}
    \nonumber
    \begin{aligned}
      y^2 &\le 2x^2 + y^2 \\
      \frac{y^2}{2x^2 + y^2} &\le 1 \\
      -x^2y &\le \frac{x^2y^3}{2x^2 + y^2} \le x^2y
    \end{aligned}
  \end{equation}
  Since
  \[
    \lim_{(x, y) \to (0, 0)} -x^2y = 0
  \]
  and
  \[
    \lim_{(x, y) \to (0, 0)} x^2y = 0
  \]
  By the Squeeze Theorem, the limit is 0 but the function is not continuous at \((0, 0)\).
\end{minipage}}

\section{Partial Derivatives}

\subsection{Partial Derivatives of Functions of Two Variables}

\begin{definition}[Partital Derivatives]
  Partital derivative of \(f\) with respect to \(x\) at \((a, b)\) is denoted by
  \[
    f_x(a, b) = g'(a)
  \]
  where
  \[
    g(x) = f(x, b)
  \]
  By the definition of derivative,
  \[
    g'(a) = \lim_{h \to 0} \frac{g(a + h) - g(a)}{h}
  \]
  and so it becomes
  \[
    f_x(a, b) = \lim_{h \to 0} \frac{f(a + h, b) - f(a, b)}{h}
  \]
  Similarly, the partial derivative of \(f\) with respect to \(y\) at \((a, b)\) is denoted by
  \[
    f_y(a, b) = \lim_{h \to 0} \frac{f(a, b + h) - f(a, b)}{h}
  \]
\end{definition}

\begin{remark}
  While finding partial derivatives, treat the other variable as a constant.
\end{remark}

\fbox{\begin{minipage}{\textwidth}
  Find the first partital derivatives
  \[
    u(r, \theta) = \sin{(r \cdot \cos{\theta})}
  \]

  With respect to \(r\)
  \begin{equation}
    \nonumber
    \begin{aligned}
      u_r &= \frac{\partial}{\partial r} \sin{(r \cdot \cos{\theta})} \\
      &= \cos{(r \cdot \cos{\theta})} \cdot \cos{\theta}
    \end{aligned}
  \end{equation}

  With respect to \(\theta\)
  \begin{equation}
    \nonumber
    \begin{aligned}
      u_{\theta} &= \frac{\partial}{\partial \theta} \sin{(r \cdot \cos{\theta})} \\
      &= -r \cdot \sin{(r \cdot \cos{\theta})} \cdot \sin{\theta}
    \end{aligned}
  \end{equation}
\end{minipage}}

\section{Tangent Planes and Linear Approximations}

\begin{formula}[Tangent Plane]
  Suppose \(f\) has continuous partial derivatives. An equation of the tangent plane to the surface \(z = f(x, y)\) at the point \(P(x_0, y_0, z_0)\) is
  \[
    z - z_0 = f_x(x_0, y_0)(x - x_0) + f_y(x_0, y_0)(y - y_0)
  \]
\end{formula}

\begin{formula}[Linear Approximation]
  The linear approximation to the function \(f\) at the point \((a, b)\) is the linear function
  \[
    L(x, y) = f(a, b) + f_x(a, b)(x - a) + f_y(a, b)(y - b)
  \]
\end{formula}

\subsection{Differentials}

\begin{definition}[Diffetentials]
  \[
    dz = f_x(x, y) dx + f_y(x, y) dy = \frac{\partial z}{\partial x} dx + \frac{\partial z}{\partial y} dy
  \]
\end{definition}

\fbox{\begin{minipage}{\textwidth}
  Find the differential of the function.
  \[
    z = x \ln{(y^2 + 1)}
  \]
  From
  \[
    dz = \frac{\partial z}{\partial x} dx + \frac{\partial z}{\partial y} dy
  \]
  Consider \(z_x\)
  \begin{equation}
    \nonumber
    \begin{aligned}
      z_x &= \frac{\partial}{\partial x} x \ln{(y^2 + 1)} \\
      &= \ln{(y^2 + 1)}
    \end{aligned}
  \end{equation}
  Consider \(z_y\)
  \begin{equation}
    \nonumber
    \begin{aligned}
      z_y &= \frac{\partial}{\partial y} x \ln{(y^2 + 1)} \\
      &= \frac{2xy}{y^2 + 1}
    \end{aligned}
  \end{equation}
  The differential is
  \[
    dz = \ln{(y^2 + 1)} dx + \frac{2xy}{y^2 + 1} dy
  \]
\end{minipage}}

\section{Higher Derivatives}

If $f$ is a function of two variables, then the second partial derivatives are defined as follows:

\begin{equation}
  \nonumber
  \begin{aligned}
    (f_x)_x &= f_{xx} = \frac{\partial}{\partial x} \left( \frac{\partial f}{\partial x} \right) &&= \frac{\partial^2 f}{\partial x^2} \\
    (f_x)_y &= f_{xy} = \frac{\partial}{\partial y} \left( \frac{\partial f}{\partial x} \right) &&= \frac{\partial^2 f}{\partial y \partial x} \\
    (f_y)_x &= f_{yx} = \frac{\partial}{\partial x} \left( \frac{\partial f}{\partial y} \right) &&= \frac{\partial^2 f}{\partial x \partial y} \\
    (f_y)_y &= f_{yy} = \frac{\partial}{\partial y} \left( \frac{\partial f}{\partial y} \right) &&= \frac{\partial^2 f}{\partial y^2}
  \end{aligned}
\end{equation}

\begin{theorem}[Claireaut's Theorem]
  Suppose \(f\) is defined on a disk \(D\) that contains the point \((a, b)\) and that the functions \(f_{xy}\) and \(f_{yx}\) are continuous on \(D\). Then
  \[
    f_{xy} = f_{yx}
  \]
\end{theorem}

\fbox{\begin{minipage}{\textwidth}
  Find all the second partial derivatives.
  \[
    f(x, y) = \ln{(ax + by)}
  \]
  Find first partial derivatives
  \begin{equation}
    \nonumber
    \begin{aligned}
      f_x &= \frac{1}{ax + by} \cdot a \\
      &= \frac{a}{ax + by} \\
      f_y &= \frac{1}{ax + by} \cdot b \\ 
      &= \frac{b}{ax + by}
    \end{aligned}
  \end{equation}
\end{minipage}}

\section{The Chain Rule}

\begin{formula}[The Chain Rule (Case 1)]
  Suppose that \(z = f(x, y)\) is differentiable function of \(x\) and \(y\), where \(x = g(t)\) and \(y = h(t)\) are both differentiable function of \(t\) and
  \[
    \frac{dz}{dt} = \frac{\partial z}{\partial x} \frac{dx}{dt} + \frac{\partial z}{\partial y} \frac{dy}{dt}
  \]
\end{formula}

\fbox{\begin{minipage}{\textwidth}
  Use the Chain Rule to find \(\frac{dz}{dt}\).
  \[
    z = \frac{x - y}{x + 2y}, x = e^{\pi t}, y = e^{-\pi t}
  \]
  Find \(\frac{dz}{dx}\) and \(\frac{dz}{dy}\)
  \begin{equation}
    \nonumber
    \begin{aligned}
      \frac{dz}{dx} &= \frac{1 \cdot (x + 2y) - (x - y) \cdot 1}{(x + 2y)^2} \\
      &= \frac{3y}{(x + 2y)^2} \\
      \frac{dz}{dy} &= \frac{-1 \cdot (x + 2y) - (x - y) \cdot 2}{(x + 2y)^2} \\
      &= \frac{-3x}{(x + 2y)^2}
    \end{aligned}
  \end{equation}
\end{minipage}}

\begin{formula}[The Chain Rule (Case 2)]
  Suppose that \(z = f(x, y)\) is differentiable function of \(x\) and \(y\), where \(x = g(s, t)\) and \(y = h(s, t)\) are both differentiable function of \(s\) and \(t\) and
  \[
    \frac{\partial z}{\partial s} = \frac{\partial z}{\partial x} \frac{\partial x}{\partial s} + \frac{\partial z}{\partial y} \frac{\partial y}{\partial s}
  \]
  and
  \[
    \frac{\partial z}{\partial t} = \frac{\partial z}{\partial x} \frac{\partial x}{\partial t} + \frac{\partial z}{\partial y} \frac{\partial y}{\partial t}
  \]
\end{formula}