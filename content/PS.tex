\chapter{Power Series}

We can write polynomial in form $P(x) = \sum_{i=0}^{n} c_{i}\cdot x^{i}$ where $c_{i}$ is coefficient and $x$ is variable.

As you can see, there is a stop position since there is $n$ in summation.

So, we can write power series in form

\[
  \sinf{n=0} c_{n}x^{n} = c_0 + c_1x + c_2x^2 + c_3x^3 + \dots
\]

where $x$ is variable, $c_{n}$ is coefficient.

\begin{theorem}[Power Series]
  If \(\sinf{n=0} c_{n}x^{n}\) converges, then we can write it as
  \[
    f(x) = c_0 + c_1x+ c_2x^2 + c_3x^3 + \dots + c_nx^n + \dots
  \]
\end{theorem}

\fbox{\begin{minipage}{\textwidth}
  How to check that \(\sinf{n=0} c_nx^n\) converges? \\
  1. Ratio test
  \begin{equation}
    \nonumber
    \begin{aligned}
      \linf{n} \Big| \frac{c_{n + 1}x^{n+1}}{c_n \cdot x^n} \Big| &= \linf{n} \Big| \frac{c_{n + 1}}{c_n} \cdot x \Big| \\
      &= |x| \cdot \linf{n} \Big| \frac{c_{n + 1}}{c_n} \Big| &&< 1
    \end{aligned}
  \end{equation}
  Let \(\alpha = \linf{n} \Big| \frac{c_{n + 1}}{c_n} \Big|\)
  \begin{equation}
    \nonumber
    \begin{aligned}
      |x| \cdot \linf{n} \Big| \frac{c_{n + 1}}{c_n} \Big| &< 1 \\
      |x| \cdot \alpha &< 1 \\
      |x| &< \frac{1}{\alpha}
    \end{aligned}
  \end{equation}
  2. Root test 
  \begin{equation}
    \nonumber
    \begin{aligned}
      \linf{n} \sqrt[n]{|c_{n}x^{n}|} &= |x| \cdot \linf{n} \sqrt[n]{|c_n|} < 1 \\
    \end{aligned}
  \end{equation}
  Let \(\alpha = \linf{n} \sqrt[n]{|c_{n}x^{n}|}\)
  \begin{equation}
    \nonumber
    \begin{aligned}
      |x| \cdot \linf{n} \sqrt[n]{|c_n|} &< 1 \\
      |x| \cdot \alpha &< 1 \\
      |x| &< \frac{1}{\alpha}
    \end{aligned}
  \end{equation}  
\end{minipage}}

\begin{definition}[General form of power series]
  \[
    \sinf{n=0} c_n(x-a)^n = c_0 + c_1(x-a) + c_2(x - a)^2 + \dots
  \]
  where \(a\) is center of power series.
\end{definition}

\begin{theorem}[Radius of Convergence]
  \[
    R = \frac{1}{\alpha} = \linf{n} \Big| \frac{c_{n + 1}}{c_n} \Big|
  \]
  where \(\alpha\) is the limit of ratio of coefficients.
\end{theorem}

\fbox{\begin{minipage}{\textwidth}
  If \(\linf{n} \Big| \frac{c_{n+1}}{c_n} \Big| = \alpha\) where \(\alpha \in \mathbb{R} - \{0\}\) \\
  then
  \[
    \sinf{n=0} c_n(x-a)^n \, \mathrm{converges} \Leftrightarrow |x-a| < \frac{1}{\alpha}
  \]
  Since \(\frac{1}{\alpha}\) is radius of convergence. \\
  So that \(x \in (a - \frac{1}{\alpha}, a + \frac{1}{\alpha})\) makes \(\sinf{n=0} c_{n}(x-a)^n\) converges. (By Ratio Test)
\end{minipage}}

We have to check that endpoints of interval of convergence.

\begin{corollary}
  There are four cases for endpoints of interval of convergence.
  \begin{itemize}
    \item \(x = a - \frac{1}{\alpha}, a + \frac{1}{\alpha}\) diverges
    \item \(x = a - \frac{1}{\alpha}\) diverges, \(x = a + \frac{1}{\alpha}\) converges
    \item \(x = a - \frac{1}{\alpha}\) converges, \(x = a + \frac{1}{\alpha}\) diverges
    \item \(x = a - \frac{1}{\alpha}, a + \frac{1}{\alpha}\) converges
  \end{itemize}
\end{corollary} 

\fbox{\begin{minipage}{\textwidth}
  If \(\linf{n} \Big| \frac{c_{n + 1}}{c_n} \Big| = 0\) \\
  Such that \(|x - a| \linf{n} \Big| \frac{c_{n + 1}}{c_n} \Big| < 1 \) converges (by Ratio Test) \\
  So that, \(\sinf{n=0} c_{n}(x - a)^n\) converges to \(\forall x \in \mathbb{R}\)
  \begin{itemize}
    \item Radius of Convergence: \(\infty\)
    \item Interval of Convergence: \((-\infty, \infty)\)
  \end{itemize}
\end{minipage}}

\fbox{\begin{minipage}{\textwidth}
  If \(\linf{n} \Big| \frac{c_{n + 1}}{c_n} \Big| = +\infty\) \\
  Such that \(|x - a| \linf{n} \Big| \frac{c_{n + 1}}{c_n} \Big| = \infty > 1 \) diverges 
  So that, \(\forall x \in \mathbb{R} - \{a\}\) diverges
  \begin{itemize}
    \item Radius of Convergence: \(0\)
    \item Interval of Convergence: \(\{a\}\)
  \end{itemize}
\end{minipage}}

\fbox{\begin{minipage}{\textwidth}
  \[
    \sinf{n=1} (-1)^nnx^n
  \]
  (Ratio Test) Consider: \(\linf{n} \Big| \frac{a_{n +1}}{a_n} \Big|\)
  \begin{equation}
    \nonumber
    \begin{aligned}
      \linf{n} \Big| \frac{a_{n + 1}}{a_n} \Big| &= \linf{n} \Big| \frac{(-1)^{n+1}(n+1)x^{n + 1}}{(-1)^nnx^n} \Big| \\
      &= \linf{n} \Big| \frac{(-1)(n+1)(x)}{n} \Big| \\
      &= |x| \linf{n} \frac{n + 1}{n} \\
      &= |x| \cdot 1 
    \end{aligned}
  \end{equation}
  Expected: \(\linf{n} \Big| \frac{a_{n +1}}{a_n} \Big| < 1\)
  \begin{equation}
    \nonumber
    \begin{aligned}
      \linf{n} \Big| \frac{a_{n +1}}{a_n} \Big| &< 1 \\
      |x| &< 1
    \end{aligned}
  \end{equation}
  Consider: When \(x = -1\)
  \begin{equation}
    \nonumber
    \begin{aligned}
      \sinf{n=1} (-1)^nnx^n &= \sinf{n=1} (-1)^nn\cdot (-1)^n \\
      &= \sinf{n=1} (-1)^{2n} \cdot n \\
      &= \sinf{n=1} n && \mathrm{(diverges)}
    \end{aligned}
  \end{equation}
  Consider: When \(x = 1\)
  \begin{equation}
    \nonumber
    \begin{aligned}
      \sinf{n=1} (-1)^nnx^n &= \sinf{n=1} (-1)^nn\cdot (1)^n \\
      &= \sinf{n=1} (-1)^{n} \cdot n && \mathrm{(diverges\, by\, Divergence\, Test)}
    \end{aligned}
  \end{equation}
  \begin{itemize}
    \item Radius of Convergence: \(1\)
    \item Interval of Convergence: \((-1, 1)\)
  \end{itemize}
\end{minipage}}

\section{Representation of Functions as Power Series}

\begin{theorem}[Sum of Geometric Series]
  \[
    \frac{1}{1 - x} = 1 + x^2 + x^3 + x^4 + \dots = \sinf{n=0} x^n \, \mathrm{where} \, |x| < 1
  \]
  Such that
  \[
    \frac{1}{1 - \Box} = \sinf{n=0} \Box^n \, \mathrm{where} \, |\Box| < 1
  \]
\end{theorem}

\fbox{\begin{minipage}{\textwidth}
  Find a power series representation for the function and determine the interval of convergence.
  \[
    f(x) = \frac{5}{1-4x^2}
  \]

  1. Write in form of power series
  \begin{equation}
    \nonumber
    \begin{aligned}
      \frac{5}{1-4x^2} &= 5 \left[ \frac{1}{1-4x^2} \right] \\
      &= 5 \sinf{n=0} (4x^2)^n
    \end{aligned}
  \end{equation}

  2. Find the radius of convergence
  
  From 1. \(|4x^2| < 1\)
  \begin{equation}
    \nonumber
    \begin{aligned}
      |4x^2| &< 1 \\
      |x^2| &< \frac{1}{4} \\
      |x| &< \frac{1}{2}
    \end{aligned}
  \end{equation}

  Interval of Convergence: \((-\frac{1}{2}, \frac{1}{2})\)
\end{minipage}}

\fbox{\begin{minipage}{\textwidth}
  Find a power series representation for the function and determine the interval of convergence.
  \[
    f(x) = \frac{4}{2x + 3}
  \]

  1. Write in form of power series
  \begin{equation}
    \nonumber
    \begin{aligned}
      \frac{4}{2x+3} &= 4 \left[ \frac{1}{2x+3} \right] \\
      &= \frac{4}{3} \left[ \frac{1}{1 + \frac{2x}{3}} \right] \\
      &= \frac{4}{3} \left[ \frac{1}{1 - (- \frac{2x}{3})}\right] \\
      &= \frac{4}{3} \sinf{n=0} \left( \frac{-2x}{3} \right)^n \\
      &= \sinf{n=0} (-1)^n \cdot \frac{2^{n+2}\cdot x^n}{3^{n+1}}
    \end{aligned}
  \end{equation}

  2. Find the radius of convergence
  
  From 1. \(\Big| \frac{-2x}{3} \Big| < 1\)
  \begin{equation}
    \nonumber
    \begin{aligned}
      \Big| \frac{-2x}{3} \Big| &< 1 \\
      \Big| x \Big| &< \frac{3}{2}
    \end{aligned}
  \end{equation}

  Interval of Convergence: \((-\frac{3}{2}, \frac{3}{2})\)
\end{minipage}}

\section{Differentiation and Integration of Power Series}

\begin{theorem}[Differentiation of Power Series]
  If \(f(x) = \sinf{n=0} c_n(x-a)^n\) and \(f(x)\) is differentiable at \(x\), then
  \[
    f'(x) = \sinf{n=1} n \cdot c_n(x-a)^{n-1}
  \]
  where \(a\) is center of power series.
\end{theorem}

\begin{theorem}[Integration of Power Series]
  If \(f(x) = \sinf{n=0} c_n(x-a)^n\) and \(f(x)\) is integrable at \(x\), then
  \[
    \int f(x) \, dx = C + \sinf{n=0} \frac{c_n}{n+1}(x-a)^{n+1}
  \]
  where \(C\) is constant and \(a\) is center of power series.
\end{theorem} 

\begin{corollary}
  We can rewrite differentiation and integration of power series as
  \[
    \frac{d}{dx} \left[ \sinf{n=0} c_{n}(x-a)^{n} \right] = \sinf{n=0} \frac{d}{dx} \left[ c_{n}(x-a)^{n} \right]
  \]
  \[
    \int \left[ \sinf{n=0} c_{n}(x-a)^{n} \right] \, dx = \sinf{n=0} \int \left[ c_{n}(x-a)^{n} \right] \, dx
  \]
\end{corollary}

\fbox{\begin{minipage}{\textwidth}
  Bessel function of order \(0\)
  \[
    J_{0}(x) = \sinf{n=0} \frac{(-1)^nx^{2n}}{2^{2n}(n!)^2}
  \]

  (a) Find the domain of \(J_{0}(x)\) \\
  By Ratio Test
  \begin{equation}
    \nonumber
    \begin{aligned}
      \linf{n} \Big| \frac{(-1)^{n+1}\cdot x^{2n+2}}{2^{n+1}\cdot ((n+1)!)^2} \cdot \frac{2^n\cdot (n!)^2}{(-1)^nx^{2n}}\Big|\ &< 1\\
      \linf{n} \Big| \frac{(-1) \cdot x^2}{2 \cdot (n+1)^2} \Big| &< 1\\
      x^2 \cdot \linf{n} \Big| \frac{1}{2(n+1)^2} \Big| &< 1\\
      0 &< 1
    \end{aligned}
  \end{equation}
  Since \(0 < 1\), \(J_{0}(x)\) converges to \(\forall x \in \mathbb{R}\)

  (b) Find the derivative of \(J_{0}(x)\) \\
  By Differentiation of Power Series
  \begin{equation}
    \nonumber
    \begin{aligned}
      J_{0}'(x) &= \sinf{n=1} n \cdot \frac{(-1)^nx^{2n-1}}{2^{2n-1}(n-1)!^2} \\
      &= \sinf{n=0} (n+1) \cdot \frac{(-1)^{n+1}x^{2n+1}}{2^{2n+1}n!^2}
    \end{aligned}
  \end{equation}
\end{minipage}}

\begin{lemma}
  \[
    \frac{1}{(1-\Box)^2} = \sinf{n=1} n \Box^{n-1} \, \mathrm{where} \, |\Box| < 1
  \]
\end{lemma}

\begin{proof}
  From Theorem 5.1.1
  \begin{equation}
    \nonumber
    \begin{aligned}
      \sinf{n=0} x^n &= \frac{1}{1 - x} \\
      &= (1 - x)^{-1}
    \end{aligned}
  \end{equation}
  Differentiate both sides
  \begin{equation}
    \nonumber
    \begin{aligned}
      \sinf{n=1} n \cdot x^{n-1} &= (1-x)^{-2} \\
      &= \frac{1}{(1-x)^2}
    \end{aligned}
  \end{equation}
  Such that, \(\frac{1}{(1-\Box)^2} = \sinf{n=1} n \Box^{n-1} \, \mathrm{where} \, |\Box| < 1 \)
\end{proof}

\begin{lemma}
  \[
    - \ln{(1-\Box)} = \sinf{n=0} \frac{\Box^{n+1}}{n+1} \, \mathrm{where} \, |\Box| < 1
  \]
\end{lemma}

\begin{proof}
  From Theorem 5.1.1
  \begin{equation}
    \nonumber
    \begin{aligned}
      \sinf{n=0} x^{n} &= \frac{1}{1 - x} \\
    \end{aligned}
  \end{equation}
  Integrate both sides
  \begin{equation}
    \nonumber
    \begin{aligned}
      \sinf{n=0} \int x^n \, dx &= \int \frac{1}{1-x} \, dx \\
      \sinf{n=0} \frac{x^{n+1}}{n+1} &= - \ln{(1-x)} \\
    \end{aligned}
  \end{equation}
  Such that, \(- \ln{(1-\Box)} = \sinf{n=0} \frac{\Box^{n+1}}{n+1} \, \mathrm{where} \, |\Box| < 1 \)
\end{proof}

\begin{lemma}
  \[
    \ln{(1+\Box)} = \sinf{n=1} (-1)^{n+1} \cdot \frac{\Box^n}{n} \, \mathrm{where} \, |\Box| < 1
  \]
\end{lemma}

\begin{proof}
  From Lemma 5.2.2
  \begin{equation}
    \nonumber
    \begin{aligned}
      - \ln{(1-\Box)} &= \sinf{n=0} \frac{\Box^{n+1}}{n+1} \\
      &= \sinf{n=1} \frac{\Box^n}{n}
    \end{aligned}
  \end{equation}
  Such that, \(\ln{(1+\Box)} = \sinf{n=1} (-1)^{n+1} \cdot \frac{\Box^n}{n} \, \mathrm{where} \, |\Box| < 1 \)
\end{proof}
  
\begin{lemma}
  \[
    \arctan{x} = \sinf{n=0} (-1)^n \cdot \frac{x^{2n+1}}{2n+1} \, \mathrm{where} \, |x| < 1
  \]
\end{lemma}

\begin{proof}
  From \(\frac{d}{dx} \arctan{x} = \frac{1}{1+x^2}\) and Theorem 5.1.1

  \begin{equation}
    \nonumber
    \begin{aligned}
      \frac{1}{1 - (-x^2)} &= \sinf{n=0} (-x^2)^n \\
      &= \sinf{n=0} (-1)^n \cdot x^{2n} \\
      \int \frac{1}{1 + x^2} &= \sinf{n=0} \int (-1)^n \cdot x^{2n} \, dx \\
      \arctan{x} &= \sinf{n=0} (-1)^n \cdot \frac{x^{2n+1}}{2n+1}
    \end{aligned}
  \end{equation}
  Such that, \(\arctan{x} = \sinf{n=0} (-1)^n \cdot \frac{x^{2n+1}}{2n+1} \, \mathrm{where} \, |x| < 1 \)
\end{proof}

\fbox{\begin{minipage}{\textwidth}
  \[
    \pi = 2\sqrt{3} \cdot \sinf{n=0} \frac{(-1)^n}{3^n(2n+1)}
  \]

  \begin{proof}
    From lemma 5.2.4
    \begin{equation}
      \nonumber
      \begin{aligned}
        \arctan{\frac{1}{\sqrt{3}}} &= \sinf{n=0} \frac{(-1)^n}{2n + 1} \cdot \left( \frac{1}{\sqrt{3}} \right)^{2n + 1}\\
        \frac{\pi}{6} &= \sinf{n=0} \frac{(-1)^n}{2n + 1} \cdot \frac{1}{3^n\sqrt{3}} \cdot \frac{\sqrt{3}}{\sqrt{3}} \\
        \pi &= 2\sqrt{3} \cdot \sinf{n=0} \frac{(-1)^n}{3^n(2n+1)}
      \end{aligned}
    \end{equation}
    Such that, \(\pi = 2\sqrt{3} \cdot \sinf{n=0} \frac{(-1)^n}{3^n(2n+1)}\)
  \end{proof}

  
\end{minipage}}