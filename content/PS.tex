\chapter{Power Series}

We can write polynomial in form $P(x) = \sum_{i=0}^{n} c_{i}\cdot x^{i}$ where $c_{i}$ is coefficient and $x$ is variable.

As you can see, there is a stop position since there is $n$ in summation.

So, we can write power series in form

\[
  \sinf{n=0} c_{n}x^{n} = c_0 + c_1x + c_2x^2 + c_3x^3 + \dots
\]

where $x$ is variable, $c_{n}$ is coefficient.

\begin{theorem}[Power Series]
  If \(\sinf{n=0} c_{n}x^{n}\) converges, then we can write it as
  \[
    f(x) = c_0 + c_1x+ c_2x^2 + c_3x^3 + \dots + c_nx^n + \dots
  \]
\end{theorem}

\fbox{\begin{minipage}{\textwidth}
  How to check that \(\sinf{n=0} c_nx^n\) converges? \\
  1. Ratio test
  \begin{equation}
    \nonumber
    \begin{aligned}
      \linf{n} \Big| \frac{c_{n + 1}x^{n+1}}{c_n \cdot x^n} \Big| &= \linf{n} \Big| \frac{c_{n + 1}}{c_n} \cdot x \Big| \\
      &= |x| \cdot \linf{n} \Big| \frac{c_{n + 1}}{c_n} \Big| &&< 1
    \end{aligned}
  \end{equation}
  Let \(\alpha = \linf{n} \Big| \frac{c_{n + 1}}{c_n} \Big|\)
  \begin{equation}
    \nonumber
    \begin{aligned}
      |x| \cdot \linf{n} \Big| \frac{c_{n + 1}}{c_n} \Big| &< 1 \\
      |x| \cdot \alpha &< 1 \\
      |x| &< \frac{1}{\alpha}
    \end{aligned}
  \end{equation}
  2. Root test 
  \begin{equation}
    \nonumber
    \begin{aligned}
      \linf{n} \sqrt[n]{|c_{n}x^{n}|} &= |x| \cdot \linf{n} \sqrt[n]{|c_n|} < 1 \\
    \end{aligned}
  \end{equation}
  Let \(\alpha = \linf{n} \sqrt[n]{|c_{n}x^{n}|}\)
  \begin{equation}
    \nonumber
    \begin{aligned}
      |x| \cdot \linf{n} \sqrt[n]{|c_n|} < 1 \\
      |x| \cdot \alpha &< 1 \\
      |x| &< \frac{1}{\alpha}
    \end{aligned}
  \end{equation}  
\end{minipage}}

\begin{definition}[General form of power series]
  \[
    \sinf{n=0} c_n(x-a)^n = c_0 + c_1(x-a) + c_2(x - a)^2 + \dots
  \]
  where \(a\) is center of power series.
\end{definition}

\begin{theorem}[Radius of Convergence]
  \[
    R = \frac{1}{\alpha} = \linf{n} \Big| \frac{c_{n + 1}}{c_n} \Big|
  \]
  where \(\alpha\) is the limit of ratio of coefficients.
\end{theorem}

\fbox{\begin{minipage}{\textwidth}
  If \(\linf{n} \Big| \frac{c_{n+1}}{c_n} \Big| = \alpha\) where \(\alpha \in \mathbb{R} - \{0\}\) \\
  then
  \[
    \sinf{n=0} c_n(x-a)^n \, \mathrm{converges} \Leftrightarrow |x-a| < \frac{1}{\alpha}
  \]
  Since \(\frac{1}{\alpha}\) is radius of convergence. \\
  So that \(x \in (a - \frac{1}{\alpha}, a + \frac{1}{\alpha})\) makes \(\sinf{n=0} c_{n}(x-a)^n\) converges. (By Ratio Test)
\end{minipage}}

We have to check that endpoints of interval of convergence.

\begin{corollary}
  There are four cases for endpoints of interval of convergence.
  \begin{itemize}
    \item \(x = a - \frac{1}{\alpha}, a + \frac{1}{\alpha}\) diverges
    \item \(x = a - \frac{1}{\alpha}\) diverges, \(x = a + \frac{1}{\alpha}\) converges
    \item \(x = a - \frac{1}{\alpha}\) converges, \(x = a + \frac{1}{\alpha}\) diverges
    \item \(x = a - \frac{1}{\alpha}, a + \frac{1}{\alpha}\) converges
  \end{itemize}
\end{corollary} 

\fbox{\begin{minipage}{\textwidth}
  If \(\linf{n} \Big| \frac{c_{n + 1}}{c_n} \Big| = 0\) \\
  Such that \(|x - a| \linf{n} \Big| \frac{c_{n + 1}}{c_n} \Big| < 1 \) converges (by Ratio Test) \\
  So that, \(\sinf{n=0} c_{n}(x - a)^n\) converges to \(\forall x \in \mathbb{R}\)
  \begin{itemize}
    \item Radius of Convergence: \(\infty\)
    \item Interval of Convergence: \((-\infty, \infty)\)
  \end{itemize}
\end{minipage}}

\fbox{\begin{minipage}{\textwidth}
  If \(\linf{n} \Big| \frac{c_{n + 1}}{c_n} \Big| = +\infty\) \\
  Such that \(|x - a| \linf{n} \Big| \frac{c_{n + 1}}{c_n} \Big| = \infty > 1 \) diverges 
  So that, \(\forall x \in \mathbb{R} - \{a\}\) diverges
  \begin{itemize}
    \item Radius of Convergence: \(0\)
    \item Interval of Convergence: \(\{a\}\)
  \end{itemize}
\end{minipage}}

\fbox{\begin{minipage}{\textwidth}
  \[
    \sinf{n=1} (-1)^nnx^n
  \]
  (Ratio Test) Consider: \(\linf{n} \Big| \frac{a_{n +1}}{a_n} \Big|\)
  \begin{equation}
    \nonumber
    \begin{aligned}
      \linf{n} \Big| \frac{a_{n + 1}}{a_n} \Big| &= \linf{n} \Big| \frac{(-1)^{n+1}(n+1)x^{n + 1}}{(-1)^nnx^n} \Big| \\
      &= \linf{n} \Big| \frac{(-1)(n+1)(x)}{n} \Big| \\
      &= |x| \linf{n} \frac{n + 1}{n} \\
      &= |x| \cdot 1 
    \end{aligned}
  \end{equation}
  Expected: \(\linf{n} \Big| \frac{a_{n +1}}{a_n} \Big| < 1\)
  \begin{equation}
    \nonumber
    \begin{aligned}
      \linf{n} \Big| \frac{a_{n +1}}{a_n} \Big| &< 1 \\
      |x| &< 1
    \end{aligned}
  \end{equation}
  Consider: When \(x = -1\)
  \begin{equation}
    \nonumber
    \begin{aligned}
      \sinf{n=1} (-1)^nnx^n &= \sinf{n=1} (-1)^nn\cdot (-1)^n \\
      &= \sinf{n=1} (-1)^{2n} \cdot n \\
      &= \sinf{n=1} n && \mathrm{(diverges)}
    \end{aligned}
  \end{equation}
  Consider: When \(x = 1\)
  \begin{equation}
    \nonumber
    \begin{aligned}
      \sinf{n=1} (-1)^nnx^n &= \sinf{n=1} (-1)^nn\cdot (1)^n \\
      &= \sinf{n=1} (-1)^{n} \cdot n && \mathrm{(diverges\, by\, Divergence\, Test)}
    \end{aligned}
  \end{equation}
  \begin{itemize}
    \item Radius of Convergence: \(1\)
    \item Interval of Convergence: \((-1, 1)\)
  \end{itemize}
\end{minipage}}