\chapter{Differential Equation}

\section{Seperable Equation}

\begin{definition}[Seperable Equation]
  A differential equation is called seperable if it can be written in the form
  \[
    \frac{dy}{dx} = g(y)h(x)
  \]
  where \(g(y)\) is a function of \(y\) only and \(h(x)\) is a function of \(x\) only.
\end{definition}

\[
  \frac{dy}{dx} = g(y)h(x)
\]
\[
  \frac{dy}{g(y)} = h(x)dx
\]
\[
  \int \frac{dy}{g(y)} = \int h(x)dx  
\]

\begin{definition}[Implicit Solution]
  An implicit solution to a differential equation is a solution that is not solved for \(y\) explicitly.
  Answer is in the form of \(F(x, y) = c\).
\end{definition}

\begin{definition}[Explicit Solution]
  An explicit solution to a differential equation is a solution that is solved for \(y\) explicitly.
  Answer is in the form of \(y = f(x)\).
\end{definition}

Sometimes there are initial condition of the equation.

\begin{theorem}[FTC 1]
  If \(f\) is continuous on \([a, b]\), then the function \(f\) defined by
  \[
    \frac{d}{dx} \int_{a}^{u(x)} f(t) dt = f(u(x))u'(x)
  \]
  is continuous on \([a, b]\) and differentiable in (a, b), and g'(x) = f(x).
\end{theorem}

\begin{definition}[Integral Equation]
  An integral equation is an equation in which an unknown function appears under an integral sign.
\end{definition}

We have to change the integral equation to a differential equation. Then, we can use FTC 1 to solve the differential equation.

\section{Linear Equation}

\begin{definition}[Linear Equation]
  A differential equation is called linear if it can be written in the form
  \[
    \frac{dy}{dx} + P(x)y = Q(x)
  \]
  where \(P(x)\) and \(Q(x)\) are continuous functions of \(x\).

\end{definition}

The term ``linear" refers to the fact that the unknown function \(y\) and its derivative \(\frac{dy}{dx}\) appear in the equation to the first power and are not multiplied together.

\begin{definition}[Integrating Factor]
  The integrating factor for the linear differential equation
  \[
    \frac{dy}{dx} + P(x)y = Q(x)
  \]
  is the function \(I(x)\) defined by
  \[
    I(x) = \exp(\int P(x)dx)
  \]
\end{definition}

We can solve the linear differential equation by the formula.
\[
  y(x) = \frac{1}{I(x)}[\int I(x)Q(x)dx + C]
\]

without any additional constraint \(c\) from integrating.

\section{Bernoulli Equation}

\begin{definition}[Bernoulli Equation]
  A differential equation is called Bernoulli if it can be written in the form
  \[
    \frac{dy}{dx} + P(x)y = Q(x)y^n
  \]
  where \(P(x)\) and \(Q(x)\) are continuous functions of \(x\).
\end{definition}

As you can see, there is $y^n$ in the equation which is not linear.

We have to change the Bernoulli equation to a linear equation by substitution.

\[
  \frac{dy}{dx} + P(x)y = Q(x)y^n
\]
divides both side by $y^n$
\[
  \frac{dy}{dx}y^{-n} + P(x)y^{1-n} = Q(x)
\]
Then we substitute $u = y^{1-n}$

\begin{equation}
  \nonumber
  \begin{aligned}
    \frac{du}{dy}&=(1-n){y^{-n}}\frac{dy}{dx} \\
    \frac{du}{dx} \cdot \frac{1}{1-n}&=y^{-n}\frac{dy}{dx}
  \end{aligned}
\end{equation}
Let $u = y^{1-n}$
\begin{equation}
  \nonumber
  \begin{aligned}
    \frac{du}{dx} &= (1-n)y^{-n}\frac{dy}{dx} \\
    \frac{du}{dx} \cdot \frac{1}{1-n} &= y^{-n} \frac{dy}{dx}
  \end{aligned}
\end{equation}
Then, try to substitute the equation with $u$.
\begin{equation}
  \nonumber
  \begin{aligned}
    \frac{du}{dx} \cdot \frac{1}{1-n} + P(x)u &= Q(x) \\
    \frac{du}{dx} + (1-n)P(x)u &= (1-n)Q(x)
  \end{aligned}
\end{equation}
Now, we can solve the equation with the linear equation method. Since the equation is linear on $u$.