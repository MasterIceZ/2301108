\chapter{Sequence}

\section{Sequence}

\begin{definition}[Sequence]
  A sequence is a function whose domain is \(\mathbb{N}\)
\end{definition}

We are considering behavior of a sequence as \(n\) becomes infinite.

\[
  \lim_{n \to \infty} a_n = \begin{cases}
    L \in \mathbb{R} & \text{convergent} \\
    \pm \infty & \text{divergent} \\
    \mathrm{more than one value} & \text{divergent}
  \end{cases}
\]

\begin{definition}[Sequence Notation]
  The sequece \(\{a_1, a_2, a_3, \dots\}\) is denoted by \(\{a_n\}\) or \(\{a_n\}_{n=1}^{\infty}\)
\end{definition}

\begin{theorem}[Limit of Sequence]
  A sequence \(\{a_n\}\) converges to \(L\).
  \[
    \lim_{n \to \infty} a_n = L
  \]
  or
  \[
    a_n \to L \; \mathrm{as} \; n \to \infty
  \]
  if for every \(\epsilon > 0\), there exists \(N\) such that
  \[
    \mathrm{if} \; n > N 
  \]
  then
  \[
    |a_n - L| < \epsilon
  \]
\end{theorem}

\begin{theorem}[]
  If \(\lim_{x \to \infty} f(x) = L\) and \(f(n) = a_n\) for all \(n \in \mathbb{N}\), then
  \[
    \lim_{n \to \infty} a_n = L
  \]
\end{theorem}

Since our function is discrete, we can use the limit of the function to find the limit of the sequence.

\begin{remark}
  We can use L'Hopital's rule to find the limit of the sequence.
\end{remark}

If subsequences of $a_{n}$ converge to different limits, then the sequence $a_{n}$ diverges. For example

\begin{equation}
  \begin{aligned}
    a_{n} &= (-1)^{n} \\
    a_{n} &= 1, -1, 1, -1, \dots
  \end{aligned}
\end{equation}

Consider: $a_{2k-1} = (-1)^{2k-1}; k \in \mathbb{N}$
\begin{equation}
  \begin{aligned}
    a_{2k-1} &= (-1)^{2k-1} \\
    a_{2k-1} &= -1
  \end{aligned}
\end{equation}

Consider: $a_{2k} = (-1)^{2k}; k \in \mathbb{N}$
\begin{equation}
  \begin{aligned}
    a_{2k} &= (-1)^{2k} \\
    a_{2k} &= 1
  \end{aligned}
\end{equation}

Since $\lim_{k \to \infty} a_{2k-1} = -1$ and $\lim_{k \to \infty} a_{2k} = 1$, the sequence $a_{n}$ diverges.

\begin{theorem}[Squeeze theorem]
  Let \(a_{n} \le b_{n} \le c_{n}\) for \(n \ge n_{0}\). \\
  If 
  \[
    \lim_{n \to \infty} a_n = \lim_{n \to \infty c_n = L}
  \]
  then
  \[
    \lim_{n \to \infty} b_{n} = L
  \]
\end{theorem}

\begin{theorem}
  If \(\lim_{n \to \infty} |a_{n}| = 0\), then \(\lim_{n \to \infty} a_{n} = 0\) 
\end{theorem}

\begin{proof}
  From $-|a_{n}| \le a_{n} \le |a_{n}|$, we have
  \[
    \lim_{n \to \infty} |a_{n}| = 0
  \]
  by the squeeze theorem, we have $\lim_{n \to \infty} a_{n} = 0$
\end{proof}

\begin{theorem}
  If \(\lim_{n \to \infty} a_{n} = L\) and the function $f$ is continuous at $L$, then
  \[
    \lim_{n \to \infty} f(a_{n}) = f(L)
  \]
\end{theorem}

\begin{remark}
  If the function is continuous,
  \[
    \lim_{n \to \infty} f(a_{n}) = f(\lim_{n \to \infty} a_{n})
  \]
\end{remark}

The sequence $\{r^{n}\}$ is converget if and only if $-1 < r \le 1$.
\[
  \lim_{n \to \infty} r^{n} = \begin{cases}
    0 & \text{if} \; -1 < r < 1 \\
    1 & \text{if} \; r = 1 \\
    \infty & \text{if} \; r > 1
  \end{cases}
\]

\section{Mathematical Induction}

\begin{definition}[Mathematical Induction]
  A proof technique used to prove a statement for all positive integers.
\end{definition}

We can use mathematical induction to find the limit of a sequence. For example

\begin{equation}
  a_{n} = \frac{1 \cdot 3 \cdot 5 \cdot \dots \cdot (2n - 1)}{n!}
\end{equation}

Consider: $a_{n} \ge (\frac{3}{2})^{n}$ and $\lim_{n \to \infty} (\frac{3}{2})^{n - 1} = \infty$ so $\lim_{n \to \infty} a_{n} = \infty$

\begin{proof}
  Proof by Mathematical Induction

  Let \(P(n)\): $a_{n} \ge (\frac{3}{2})^{n}$ for \(n \in \mathbb{N}\)

  \textbf{Base Case:} n = 1
  \[
    a_{1} = \frac{1}{1!} = 1
  \]
  \[
    (\frac{3}{2})^{1 - 1} = (\frac{3}{2})^0 = 1
  \]
  Since $a_{1} \ge (\frac{3}{2})^{0}$, \(P(1)\) is true.

  \textbf{Inductive Step:} Assume \(P(k)\) is true for \(k \in \mathbb{N}\)
  \[
    a_{k} \ge a_{k} \cdot \frac{2(k + 1) - 1}{k + 1} = a_k \cdot \frac{2k + 1}{k + 1}
  \]
  \[
    (\frac{3}{2})^k = (\frac{3}{2})^{k - 1} \cdot \frac{3}{2}
  \]

  Consider
  \[
    a_{k} \cdot \frac{2k + 1}{k + 1} \ge (\frac{3}{2})^{k - 1} \cdot \frac{3}{2}
  \]
  So \(P(k) \to P(k + 1)\) is true. Such that \(P(n)\) is true for all \(n \in \mathbb{N}\).
\end{proof}

\section{Monotonic and Bounded Sequences}

Our goal is to determine if a sequence is convergent or divergent. We can tell that a sequence is "convergent" but may not know the absoulte value of the limit.

\begin{definition}[Increasing Sequence]
  A sequence \(\{a_{n}\}\) is increasing if \(a_{n} < a_{n + 1}\) for all \(n \ge 1\).
\end{definition}

\begin{definition}[Decreasing Sequence]
  A sequence \(\{a_{n}\}\) is decreasing if \(a_{n} > a_{n + 1}\) for all \(n \ge 1\).
\end{definition}

\begin{definition}[Monotonic Sequence]
  A sequence \(\{a_{n}\}\) is monotonic if it is either increasing or decreasing.
\end{definition}

\begin{remark}
  In this case, increasing and decreasing sequences are defined by "strictly" increasing and decreasing in order.
\end{remark}

\begin{definition}[Bounded Sequence]
  A sequence \(\{a_{n}\}\) is bounded above if there is a number \(M\) such that
  \[
    a_{n} \le M \; \mathrm{for} \; n \ge 1
  \]
  A sequence \(\{a_{n}\}\) is bounded below if there is a number \(m\) such that
  \[
    a_{n} \ge m \; \mathrm{for} \; n \ge 1
  \]
  If a sequence is bounded above and below, then it is called a bounded sequence.
\end{definition}

\begin{theorem}[Monotonic Sequence Theorem]
  Every bounded, monotonic sequence is convergent.
\end{theorem}

\begin{remark}
  Techniques to check that the sequence is increasing or decreasing:
  \begin{itemize}
    \item \(a_{n + 1} - a_{n}\) \begin{itemize}
      \item If \(a_{n + 1} - a_{n} > 0\), then the sequence is increasing
      \item If \(a_{n + 1} - a_{n} < 0\), then the sequence is decreasing
    \end{itemize}
    \item \(\frac{a_{n + 1}}{a_n}\) \begin{itemize}
      \item If \(\frac{a_{n + 1}}{a_n} > 1\), then the sequence is increasing
      \item If \(\frac{a_{n + 1}}{a_n} < 1\), then the sequence is decreasing
    \end{itemize}
    \item Find $f'(n)$ \begin{itemize}
      \item If $f'(n) > 0$, then the sequence is increasing
      \item If $f'(n) < 0$, then the sequence is decreasing
    \end{itemize}
  \end{itemize}
\end{remark}

Example: Prove that the following sequence is convergent
\[
  a_{n} = 1 + \frac{1}{2!} + \frac{1}{3!} + \dots + \frac{1}{n!} \; \mathrm{for} \; n \in \mathbb{N}
\]

\begin{proof}
  Since there are factorials in each term, we cannot use the derivative to determine if the sequence is increasing or decreasing.

  1. Determine if the sequence is increasing or decreasing
  \[
    \begin{aligned}
      a_{n + 1} - a_{n} &= (1 + \frac{1}{2!} + \frac{1}{3!} + \dots + \frac{1}{(n + 1)!}) - (1 + \frac{1}{2!} + \frac{1}{3!} + \dots + \frac{1}{n!}) \\
      &= \frac{1}{(n + 1)!} 
    \end{aligned}
  \]

  Since \(\frac{1}{(n + 1)!} > 0\), the sequence is increasing.

  2. Determine if the sequence is bounded
  \[
    a_{n} = 1 + \frac{1}{2!} + \frac{1}{3!} + \dots + \frac{1}{n!} \le 1 + \frac{1}{2} + \frac{1}{2^2} + \frac{1}{2^3} + \dots + \frac{1}{2^{n - 1}} + \dots
  \]

  \[
    \begin{aligned}
      \sum_{i=1}^{n} \frac{1}{i!} &\le \sum_{i=1}^{\infty} (\frac{1}{2})^{i - 1}
      &\le \frac{1}{1 - \frac{1}{2}} = 2
    \end{aligned}
  \]

  So \(1 \le a_{n} \le 2\). The sequence is bounded.

  By thorem: \(\{a_n\}\) is monotonic and bounded, so it is convergent.
\end{proof}

\begin{theorem}
  If there are 2 or more subsequences of \(\{a_{n}\}\) which converges to different values then \(\{a_{n}\}\) is divergent.
\end{theorem}

For example: $\lim_{n \to \infty} |a_{n}| \neq 0$ and there is $(-1)^{n}$ in each term.

\[
  \cos{(n\pi)} = \begin{cases}
    a_{2k} = 1 & \text{if} \; n = 2k \\
    a_{2k - 1} = -1 & \text{if} \; n = 2k - 1
  \end{cases}
\]

A sequence $\{a_{n}\}$ is given by $a_{1} = \sqrt{2}$, $a_{n + 1} = \sqrt{2 + a_{n}}$ for $n \ge 1$.

(a) Find $\lim_{n \to \infty} a_{n}$

Let $L = \lim_{n \to \infty} a_{n}$

\[
  \begin{aligned}
    \lim_{n \to \infty} a_{n} &= \lim_{n \to \infty} \sqrt{2 + a_{n}} \\
    L &= \sqrt{2 + L} \\ 
    L^2 &= 2 + L \\
    L^2 - L - 2 &= 0 \\
    (L - 2)(L + 1) &= 0 \\
    L &= 2, -1
  \end{aligned}
\]

Since under the square root, $a_{n} \ge 0$, so $L = 2$.

(b) Prove that $\{a_{n}\}$ is increasing and bounded above by 3. Then use the Monotonic Sequence Theorem to show that $\{a_{n}\}$ is convergent.

\begin{proof}
  Let \(P(n)\): \(a_{n + 1} > a_{n}\) and $a_{n} \le 3$

  \textbf{Base Case: } n = 1
  \[
    a_1 = \sqrt{2} \; \mathrm{and} \; a_2 = \sqrt{2 + \sqrt{2}}
  \]
  Since \(\sqrt{2} < \sqrt{2 + \sqrt{2}}\) and \(\sqrt{2} \le 3\), \(P(1)\) is true.

  \textbf{Inductive Step: } Assume \(P(k)\) is true for \(k \in \mathbb{N}\)
  \[
    a_{k + 1} > a_{k} \; \mathrm{and} \; a_{k} \le 3
  \]
  \[
    a_{k + 2} = \sqrt{2 + a_{k + 1}}
  \]
  Since \(a_{k + 1} > a_{k}\), we have
  \[
    \sqrt{2 + a_{k + 1}} > \sqrt{2 + a_{k}}
  \]
  From $a_{k} \le 3$, we have
  \[
    \begin{aligned}
      a_{k + 1} &= \sqrt{2 + a_{k}} \\
      \sqrt{2 + a_{k}} &\le \sqrt{2 + 3} \\
      a_{k + 1} &\le \sqrt{5} \\
      &< \sqrt{9} \\
      &< 3
    \end{aligned}
  \]
  So \(P(k) \to P(k + 1)\) is true. Such that \(P(n)\) is true for all \(n \in \mathbb{N}\).
\end{proof}