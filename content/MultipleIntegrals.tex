\chapter{Multiple Integrals}

\section{Double Integrals over Rectangles}

\subsection{Volume and Double Integrals}

In a similar way to how we defined the definite integral of a function of one variable as the limit of the sum of the areas of rectangles, we can define the definite integral of a function of two variables over a region in the plane as the limit of the sum of the volumes of rectangular boxes.

\[
  R = [a, b] \times [c, d] = \{(x, y) \in \mathbb{R}^2 | a \le x \le b, c \le y \le d\}
\]

We have function $f(x, y) \ge 0$. The graph of $f$ is a surface with equation $z = f(x, y)$. Let $S$ be the solid that lies above $R$ and under the graph of $f$, that is,
\[
  S = \{(x, y, z) \in \mathbb{R}^3 | 0 \le z \le f(x, y), (x, y) \in R\}
\]

\begin{definition}[Double Integral]
  The double integral of \(f\) over the rectangle \(R\) is
  \[
    \iint_R f(x, y) \, dA = \lim_{m, n \to \infty} \sum_{i=1}^{m} \sum_{j=1}^{n} f(x_{ij}^*, y_{ij}^*) \Delta A
  \]
  if the limit exists.
\end{definition}

\subsection{Iterated Integrals}

Suppose that \(f\) is a function of two variables that is integrable over the rectangle \(R = [a, b] \times [c, d]\). We can evaluate the double integral of \(f\) over \(R\) by evaluating two single integrals.

\[
  A(x) = \int_c^d f(x, y) \, dy
\]

If we now integrate \(A(x)\) from \(a\) to \(b\), we get the double integral of \(f\) over \(R\).

\[
  \int_a^b A(x) \, dx = \int_a^b \left( \int_c^d f(x, y) \, dy \right) \, dx = \iint_R f(x, y) \, dA
\]

\begin{theorem}[Fubini's Theorem]
  If \(f\) is continuous on the rectangle
  \[
    R = \{(x, y) | a \le x \le b, c \le y \le d\}
  \]
  then,
  \[
    \iint_R f(x, y) \, dA = \int_a^b \left( \int_c^d f(x, y) \, dy \right) \, dx = \int_c^d \left( \int_a^b f(x, y) \, dx \right) \, dy
  \]
\end{theorem}

If \(f(x, y) \ge 0\), then the volume \(V\) of the solid that lies above the rectangle \(R\) and below the surface \(z = f(x, y)\) is
\[
  V = \iint_R f(x, y) \, dA
\]

\section{Double Integrals over General Regions}
\[
  F(x, y) = \begin{cases}
    f(x, y) & \text{if } (x, y) \in D \\
    0 & \text{if } (x, y) \notin D
  \end{cases}
\]
If $F$ is integrable over $R$, then we define the double integral of $f$ over $D$ by

\begin{formula}
  \[
    \iint_{D} f(x, y) \, dA = \iint_{R} F(x, y) \, dA
  \]
  where \(F\) is given by the formula above.
\end{formula}

A plane region \(D\) is called \textbf{simple} if it is bounded and can be expressed as
\[
  D = \{(x, y) | a \le x \le b, g_1(x) \le y \le g_2(x)\}
\]

\begin{definition}
  If \(f\) is continuous on a simple region \(D\) described by
  \[
    D = \{(x, y) | a \le x \le b, g_1(x) \le y \le g_2(x)\}
  \]
  then
  \[
    \iint_{D} f(x, y) \, dA = \int_{a}^{b} \int_{g_1(x)}^{g_2(x)} f(x, y) \, dy \, dx
  \]
\end{definition}

We also consider plane regions of the form \(D = \{(x, y) | c \le y \le d, h_1(y) \le x \le h_2(y)\}\).

\begin{definition}
  If \(f\) is continuous on a simple region \(D\) described by
  \[
    D = \{(x, y) | c \le y \le d, h_1(y) \le x \le h_2(y)\}
  \]
  then
  \[
    \iint_{D} f(x, y) \, dA = \int_{c}^{d} \int_{h_1(y)}^{h_2(y)} f(x, y) \, dx \, dy
  \]
\end{definition}

\fbox{\begin{minipage}{\textwidth}
  \begin{equation}
    \nonumber
    \begin{aligned}
      \int_{0}^{2} \int_{0}^{y^2} x^2y \, dx \, dy &= \int_{0}^{2} \left[ \int_{0}^{y^2} x^2y \, dx \right] \, dy \\
      &= \int_{0}^{2} y \int_{0}^{y^2} x^2 \, dx \, dy \\
      &= \int_{0}^{2} y \left[ \frac{x^3}{3} \Big|_{0}^{y^2} \right] \\
      &= \frac{1}{3} \int_{0}^{2} y \cdot y^6 \, dy \\
      &= \frac{1}{3} \int_{0}^{2} y^7 \, dy \\
      &= \frac{1}{3} \left[ \frac{y^8}{8} \Big|_{0}^{2} \right] \\
      &= \frac{1}{3} \cdot \frac{2^8}{8} \\
      &= \frac{2^5}{3} = \frac{32}{3}
    \end{aligned}
  \end{equation}
\end{minipage}}

\subsection{Changing the Order of Integration}

Fubini's Theorem tell us that we can express a double integral as an iterated integral in two different orders.
Sometimes one order is much easir to evaluate than the other or it may be the only way to evaluate the integral.

\begin{remark}
  This method require us to draw the region of integration.
\end{remark}

\fbox{\begin{minipage}{\textwidth}
  Evaluate the integral by reversing the order of integration.
  \[
    \int_{0}^{1} \int_{3y}^{3} e^{x^2} \, dx \, dy
  \]
  \begin{equation}
    \nonumber
    \begin{aligned}
      \int_{0}^{1} \int_{3y}^{3} e^{x^2} \, dx \, dy &= \int_{0}^{3} \int_{0}^{x / 3} e^{x^2} \, dy \, dx \\
      &= \int_{0}^{3} e^{x^2} \cdot y \Big|_{y=0}^{y=x/3} \, dx \\
      &= \int_{0}^{3} e^{x^2} \cdot \frac{x}{3} \, dx \\
      &= \frac{1}{3} \int_{0}^{3} x \cdot e^{x^2} \, dx
    \end{aligned}
  \end{equation}
  Let \(u = x^2\), then \(du = 2x \, dx\)
  \begin{equation}
    \nonumber
    \begin{aligned}
      \frac{1}{3} \int_{0}^{3} x \cdot e^{x^2} \, dx &= \frac{1}{3} \int_{0}^{9} x \cdot e^{u} \, \frac{du}{2x} \\
      &= \frac{1}{6} \int_{0}^{9} e^{u} \, du \\
      &= \frac{1}{6} e^{u} \Big|_{0}^{9} \\
      &= \frac{1}{6} (e^9 - 1)
    \end{aligned}
  \end{equation}
\end{minipage}}

\section{Applications of Double Integrals}

\subsection{Density and Mass}

\begin{formula}[Mass]
  Mass = Density \(\times\) Volume
  \[
    m = \lim_{k, l \to \infty} \sum_{i=1}^{k} \sum_{j=1}^{l} \rho(x_{ij}^*, y_{ij}^*) \Delta A = \iint_{D} \rho(x, y) \, dA
  \]
\end{formula}

\subsection{Moments and Centers of Mass}

\begin{formula}[Moments]
  The moment about the \(y\)-axis is
  \[
    M_y = \iint_{D} x \rho(x, y) \, dA
  \]
  The moment about the \(x\)-axis is
  \[
    M_x = \iint_{D} y \rho(x, y) \, dA
  \]
\end{formula}

\begin{formula}[Center of Mass]
  The \(x\)-coordinate of the center of mass is
  \[
    \bar{x} = \frac{M_y}{m}
  \]
  The \(y\)-coordinate of the center of mass is
  \[
    \bar{y} = \frac{M_x}{m}
  \]
\end{formula}
